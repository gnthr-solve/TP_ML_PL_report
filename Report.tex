%Basic Formatting and Packages---------------------------------------------------------------------------------------------------------------------------------------------------------------------------
\documentclass[11pt,a4paper]{article}
\usepackage{amsmath} 
\usepackage{amssymb}   
\usepackage{graphicx}
\usepackage{geometry}
\usepackage{mathtools}
\usepackage[dvipsnames]{xcolor}
\usepackage{enumitem}
\usepackage{tcolorbox}
\usepackage{verbatim}
\usepackage{subcaption}
\usepackage[ruled, noend]{algorithm2e}

\usepackage{tikz}
\usetikzlibrary{shapes.geometric, arrows, positioning}

\geometry{left=2cm,right=2cm,top=2.5cm,bottom=2.5cm}


%Own Commands----------------------------------------------------------------------------------------------------------------------------------------------------------------------------------------------------
%1. Inequality in set: ieset
%Default: lhs is tau
\newcommand{\ieset}[2][\tau]{\{#1 \le #2\}} 
%Example: to get {X <= t} use this expression $\ieset[X]{t}$

%2. Conditional Expectation for Sigma Algebra: scEx
%Default: sigma Algebra is F_n
\newcommand{\scEx}[2][n]{E[#2 \mid \mathfrak{F}_#1]}
%Example: to get E[ X | F_tau ] use this expression $\scEx[\tau]{X}$

%3. In curly brackets: icb
\newcommand{\icb}[1]{\{ #1 \}}
%Example: to get {a = b} use this expression $\icb{a = b}

%4. Norm: norm
\newcommand{\norm}[1]{\left\lVert#1\right\rVert}
%Example: to get ||X|| use the expression \norm{X}

%5. Column Vector
\newcommand*\colvec[1]{\begin{pmatrix}#1\end{pmatrix}}
%Example: \colvec{a \\ b} gives vector (a,b) transposed.

%6.Row Vector
\newcommand{\rvec}[1]{\begin{bmatrix} #1 \end{bmatrix}}
%Example: \rvect{a & b} gives vector (a,b)


%Own Operators----------------------------------------------------------------------------------------------------------------------------------------------------------------------------------------------------
\DeclareMathOperator*{\argmax}{arg\,max}
\DeclareMathOperator*{\argmin}{arg\,min}



\title{\textbf{The influence of distribution characteristics and data balancing on classification bias in highly unbalanced data sets}}
    \author{
        \parbox{\linewidth}{\centering
            Zekiye Erarslan, Manuel Günther, Artemii Redkin, Matteo Zannini
        }
     }
    \date{30.09.2023}


\begin{document}
%Inputs from content----------------------------------------------------------------------------------------------------------------------------------------------------------------------------------------------------
\section{Introduction}

In recent years, the field of machine learning has evolved from an emerging science into a widely applied technology, finding applications across various domains such as business, industry, and scientific research. However, as machine learning techniques have gained prominence, a critical challenge has surfaced known as the class imbalance problem. This problem, characterized by a significant disparity in the number of samples between different classes within a dataset, has profound implications for classification performance and decision-making in real-world applications.

It has become increasingly apparent that imbalanced datasets can lead to suboptimal classification results, prompting researchers to explore solutions to mitigate its impact. 
The class imbalance problem is widespread and it is essential to understand that class imbalances can manifest in various application domains, including fraud detection, risk management, text classification, and medical diagnosis. In some cases, these imbalances are inherent to the problem, while in others, they arise due to limitations in data collection processes or the need for human intervention in selecting examples for training.  For instance, suppose the task is to determine whether a patient has cancer. If we train a model on data with such a ratio of classes, it will favor the majority class - that is, it will make a prediction that the patient has cancer, which can lead to very serious consequences.

Both data-level and algorithmic-level solutions have been proposed to address class imbalance. These include resampling techniques such as oversampling and undersampling, adjustments to class-specific costs, threshold tuning, and recognition-based learning approaches. Researchers have dedicated considerable effort to developing and refining these methods, aiming to improve the performance and fairness of machine learning models in the face of imbalanced data.

In this report, we illustrate a comparative analysis aiming to highlight the influence of data balancing and other factors in the process of classifying imbalanced datasets. In addition to assessing the performance of established data-level balancing algorithms, our objective includes the exploration of diverse classification scenarios wherein various factors may exert an influence on classification efficacy.

The report is organized into distinct sections, including Methods, Results, and Discussion. Within the Methods section, we comprehensively elucidate our approach, which comprises a four-step pipeline encompassing data generation, balancing, classification, and output analysis. It's noteworthy to emphasize that each step of this methodology is supported by extensive literature research, ensuring a robust foundation for our approach.

In the Results section, we provide the most significant findings derived from the systematic variation of parameters within our defined pipeline. Through rigorous experimentation, we explore the impact of parameter adjustments on the classification outcomes. To facilitate a comprehensive understanding of our findings, we employ data visualization techniques and provide insightful plots that vividly represent the observed results. 

In the final section, the Discussion, we delve into a thorough analysis of the results and draw meaningful conclusions from our study. We closely examine what our findings mean in practical terms and how they can be applied in real-world situations. Additionally, we consider areas where our study could be improved or enhanced for future research. By acknowledging both the strengths and limitations of our work, we lay the groundwork for future studies to build upon our findings and advance the field of classifying imbalanced datasets.

\section{Methods}

\subsection{Data balancing methods}

\subsection{Classifiers}

\subsection{Output analysis metrics}

\subsection{Pipeline}

\subsection{Data balancing}

A data balancer attempts to close the gap between the number of samples in the majority class and the ones in the minority class.
This can be done by undersampling the majority class, oversampling the minority class or creating artificial samples for the minority class.
There is a fundamental problem apparent when thinking of how the latter can be accomplished:
If we knew the characteristics that distinguish the minority class from the majority class so well, that we could reliably create samples from it,
then the entire classification step would in a way be redundant as we could already assert class membership by those characteristics.
For this reason, SMOTE and other synthetic sampling methods can only produce imperfect representations of the real minority distributions.

With this in mind here are some important characteristics that we would hope a useful balancer can fulfill:
\begin{enumerate}[label=$\bullet$]
\item The balancer should be generalisable and applicable to different levels of imbalance and subsequent classification methods;
\item It should maintain the actual class structure and represent the minority class pattern accurately;
\item It should support a correct class identification.
\end{enumerate}
   
We hereby summarise the main features of the data-level balancing algorithms that we implemented in the pipeline. Typically, these methods are categorised into three main groups: synthetic samplers, resamplers and hybrid samplers.

\subsubsection{SMOTE}
\textbf{SMOTE} (Synthetic Minority Oversampling Technique) is probably the most known of the balancing methods. 
SMOTE is an over-sampling technique which expands the training set $\mathcal{D}_\text{train}$ by generating synthetic minority class samples.

Given the parameter $k$, at every iteration the algorithm does the following steps:

\begin{enumerate}[label=(\roman*)]
\item Select a random sample $(x_j^1, y_j^1)$ of the minority class
\item For the feature vector $x_j^1$ find the $k$ nearest minority neighbours $x_{l_1}^1 \dots x_{l_k}^1$
\item Select one of these neighbours at random, say $x_{l}^1$ and sample a uniform $u \sim \mathcal{U}(0,1)$
\item Add $(x, 1)$ with $x = x_j^1 + u (x_j^1 - x_l^1)$ to the training set
\end{enumerate}

until the desired number of minority samples is achieved.
This process creates new data points along line segments between minority feature vectors, effectively making the decision boundary of the minority class more inclusive. 
We decided to include SMOTE in our study as it is a reference method, even though important drawbacks have been detected in previous articles, 
such as presenting computational complexity quadratic in the size of the minority class and distortion of its distribution due to localisation of the selected target points.



\begin{comment}
	\begin{figure}[H]
	\label{fig:tube}
	\centering
	\includegraphics[height=4.0cm]{tube.jpg}
	\caption{Conservation in thin long tube (3D), with variables varying only in one dimension (from C.P. Fall, 2005)}
	\end{figure}
\end{comment}


\subsubsection{Borderline SMOTE}

Borderline-SMOTE is a specialized technique designed to handle imbalanced datasets. It focuses on oversampling examples near class boundaries, which are crucial for accurate classification. Unlike traditional oversampling methods that generate synthetic examples for the entire minority class, Borderline-SMOTE specifically targets these borderline examples within the minority class.

This method determines if a minority class instance is eligible for oversampling with the SMOTE technique based on the majority class representation among its nearest neighbors. By doing so, Borderline-SMOTE ensures a more precise and effective oversampling strategy.

Furthermore, implementing specific variants like Borderline-SMOTE I or Borderline-SMOTE II can be advantageous, especially in cases with outliers or sparsely scattered minority observations. These variants provide an extra layer of refinement to the oversampling process, enhancing the algorithm's performance.

In essence, Borderline-SMOTE's targeted approach, coupled with its careful consideration of instance eligibility, equips it to strengthen the learning process, particularly in regions near class boundaries where accurate classification is most critical. By incorporating this technique, imbalanced datasets can undergo a tailored oversampling treatment, ultimately improving the model's ability to navigate complex class boundaries~\cite{Nguyen2009,Gupta2018,Brandt2020}.


%\subsubsection{K-means SMOTE}


\subsubsection{ADASYN} is another synthetic sampler 
The ADASYN algorithm addresses class imbalance by generating synthetic observations tailored to the learning difficulty of specific minority class instances. It focuses on producing more synthetic data for minority observations that pose a greater challenge for the model to learn. This approach is similar to SMOTE, as it also creates synthetic observations for the minority class. However, ADASYN distinguishes itself by prioritizing the generation of synthetic data for instances that are comparatively harder to learn within the given model.

It generates additional synthetic observations for minority class instances located in regions where there are more majority class observations within the k-nearest neighbors' range. Conversely, if a minority observation has no majority class neighbors within this range, no synthetic data is generated for that particular instance. The rationale behind this lies in the understanding that instances without nearby majority class samples are inherently more challenging to learn than those situated further from majority class observations.

Fundamentally, ADASYN's approach is rooted in the adaptive generation of minority data samples, tailoring the synthesis process to the distribution and learning difficulty of the minority class~\cite{Brandt2020, He2008}.









\begin{comment}
% I put this algorithm here for reference on how algorithm2e is used
\NoCaptionOfAlgo
\begin{algorithm}[H]
\SetAlgoLined
\DontPrintSemicolon
\SetKwComment{Comment}{$\triangleright$\ }{}
\SetAlCapSkip{1em}
\SetAlCapNameFnt{\normalfont\normalsize}
%\TitleOfAlgo{Timestep $\textcolor{darkgray}{t \to t + \delta t}$}
\caption{Timestep $\textcolor{darkgray}{t \to t + \delta t}$}

Select a random sample of the minority class\;

\For{c in \texttt{cells}}{
    $
    \textcolor{darkgray}{r_c} :=
    \begin{cases*}
        \color{teal}{r_S} &\text{if c is sensitive} \\
        \color{purple}{r_R} &\text{if c is resistent}\\
    \end{cases*}$\;
    Total propensity $\textcolor{darkgray}{p := (r_c + d_T) \delta t}$\;
    \uIf{$\textcolor{Purple}{u < p}$}{ \tcp{cell active}
        \uIf{$\textcolor{Purple}{u < \frac{r_c \delta t}{p}}$}{
            \texttt{AttemptProliferation}\;
        }
        \Else{
            \texttt{Death}\;
        }
    } 
    \Else{  \tcp{cell inactive}
        \texttt{continue}\;
    }
}
\end{algorithm}
\end{comment}

\input{content/Evaluating a New Marker for Risk Prediction Using the Test Tradeoff}
\input{content/harm_of_imbal_correction}

\subsection{Metrics}

This section gives a brief overview of the metrics commonly used to assess the performance of a classifier in a binary classification situation.
Given the true y-values of a test set and classifiers predictions on the corresponding feature vectors most metrics are based on the so called \textbf{confusion matrix}.

\begin{figure}[H]
	\centering
  	\includegraphics[width=0.75\linewidth]{assets/confusion_matrix.png}
  	\captionof{figure}{Confusion Matrix from wikipedia}
  	\label{fig:confusion_matrix}
\end{figure}

It summarises performance with the number of true positves (TP), true negatives (TN), false positives (FP) and false negatives (FN). 
What is a positive depends on the target class. In our case this is the class $1$ (the minority class) while a negative corresponds to class $0$ (the majority class).

From these values many standard measures are derived:\\

\textbf{Accuracy:}
Accuracy is the ratio between all that has been labeled correctly and the total number of samples, i.e.
\[
	\text{acc} = \frac{TP + TN}{TP + TN + FP + FN}
\]
It is a commonly used but fairly basic measure that often fails to accurately represent a classifiers performance in an imbalanced situation,
as for a high class imbalance the accuracy is high for a classifier that stubbornly predicts majority class for every sample.\\

\textbf{Sensitivity / True Positive Rate / Recall:}
Recall reflects the fraction of positive samples the classifier has identified i.e.
\[
	\text{rec} = \frac{TP}{TP + FN}
\]
It is more relevant in the imbalanced case as a classifier that chooses to label all before it as majority class to obtain good accuracy will have low recall.\\

\textbf{Precision:}
Precision represents the fraction of samples that have been correctly labeled positive i.e.
\[
	\text{prec} = \frac{TP}{TP + FP}
\]
This measure allows to assess whether the classifier tends to be correct when labelling a sample as minority class.\\

\textbf{False Positive Rate:}
FPR gives the fraction of negative samples that are still incorrectly classified as positive.
\[
	\text{FPR} = \frac{FP}{TN + FP}
\]
It is used to compute the receiver operator characteristic and relates to the next measure.\\

\textbf{Specificity:}
Gives the fraction of samples classified as correctly negative versus all negatives.
\[
	\text{spec} = \frac{TN}{TN + FP} = 1 - \text{FPR}
\]
Specificity is especially important when assessing the frequency and potential cost of false positives.\\


We have already described four different measures based on the confusion matrix, each of them assessing different aspects of the performance of a classifier.
But since all of these contribute important information, and since a classifier can have good scores in one but bad scores in another, 
how can one give a unified answer to the question "How good is my classifier?" when it comes to evaluation and decision making?
This question is the idea for the measures that follow.\\

One measure that intends to combine at least two of the metrics mentioned above is the \textbf{F1-score}.
It is the harmonic mean of precision and recall, where the harmonic mean for a set of positive real numbers $x_1, \dots, x_n$ is given by
\[
	H(x_1, \dotsm x_n) = \frac{n}{\sum_{j=1}^n \frac{1}{x_j}}
\]
which applied to precision and recall becomes
\[
	F = \frac{2}{ \frac{1}{\text{rec}} + \frac{1}{\text{prec}} } = 2 \frac{\text{rec} \, \text{prec}}{ \text{rec} + \text{prec} }.
\]
There are also weighted versions of the F1-score that are supposed to take into account whether recall or precision are more important in a given situation.\\

Given the test set, the classifiers we used in our project predict the probabilities of class membership for each sample.
The standard way in which they then map these probabilities to a class prediction is by simply predicting the class with the largest probability.
In practical application this approach may be flawed however. 
For a good classifier the predicted probability, which represents the confidence it has that a new sample belongs in the respective class, 
should closely match the actual probability of the class given the samples features. 
When a classifiers decision involves risk a user might want select a specific probability threshold for a positive prediction. \\

%Another important measure that provides a more comprehensive assessment of a classifiers performance is the 
\textbf{Receiver Operator Characteristic (ROC)}
ROC curves take this issue into account, by assessing the performance of a classifier across a range of probability thresholds.
On a theoretical level, supposing here $X$ is the random variable representing the distribution of the feature vectors 
and our classifier outputs a continuous probability score $s(X)$, the ROC is obtained by considering different cutoff thresholds for that score.
Suppose that if $s(X) > \tau$, we predict the positive class, then the true positive rate (TPR) and the false positive rate (FPR) are given by
\[
    	\begin{aligned}
    		TPR(\tau) &= \mathbb{P}(s(X) > \tau | Y = 1) \\
    		FPR(\tau) &= \mathbb{P}(s(X) > \tau | Y = 0)
    	\end{aligned}
\]
where $Y$ is the r.v. representing the true class of $X$ for every realisation. 
%In this theoretical context we can assign a new meaning to two measures we saw earlier:
%TPR is the probability that the classifier ranks a randomly chosen positive instance higher than a randomly chosen negative instance. 
%Similarly, FPR is the probability that the classifier ranks a randomly chosen negative instance higher than a randomly chosen positive instance.

The ROC curve plots $TPR(\tau)$ against $FPR(\tau)$ for all possible thresholds $\tau$, producing a curve that ranges from $(0,0)$ to $(1,1)$.
We can interpret a ROC plot as plotting the path of a function
\[
	f: \mathbb{R} \to [0,1]^2, \quad  \tau \mapsto (TPR(\tau), FPR(\tau))
\]

\begin{figure}[H]
  	\centering
  	\includegraphics[width=0.5\linewidth]{assets/Roc_curve.png}
  	\captionof{figure}{Illustration of a ROC curve from wikipedia by MartinThoma}
  	\label{fig:roc_curve}
\end{figure}

The \textbf{Area Under the ROC Curve (AUC)} is, as the name suggests, the integral of the area under the ROC curve of a model
It provides a single scalar value that represents the expected performance of the classifier under a variety of cutoff thresholds.
An AUC of $1$ indicates perfect discrimination, while an AUC of $0.5$ indicates a prediction performance no better than random chance.

AUC can also be interpreted in terms of the probability that the classifier will rank a randomly chosen positive instance higher than a randomly chosen negative instance,
assuming that one positive and one negative instance are chosen at random \cite{hanley1982meaning}.

In practice ROC curves and AUC values are obtained by sorting the classified samples by score thus AUC is essentially a rank order statistic.

\textbf{Calibration}
We mentioned before that a classifiers probability predictions should in a perfect world correspond to the actual class membership probability of a feature vector.
Expressed theoretically again this would mean for the probability function $p$ of the classifier that given a new feature sample $x_\text{new}$
\[
	p(x_\text{new}) = \mathbb{P}(Y = 1 | X = x_\text{new})
\]
Of course the conditional probability on the right hand side of this equation is unknown in practice and thus this condition cannot be directly evaluated.
Calibration curves make some simplifications to still have some measure of how good the probability predictions of a model are.
Instead of matching the classifiers prediction with the conditional probability a calibration curve is created by comparing mean predicted probabilities 
with mean frequency of occurrence on a test set.
This is done by firstly sorting the predicted probabilities from lowest to highest and aligning them with the corresponding true class value.
Then the aligned lists are binned and for each bin the mean predicted probability and the mean frequency of class $1$ occurrence are calculated.
One then plots, either directly or after interpolation, the predicted probability versus the mean frequency.
For details on the implementation please check the corresponding methods in the \texttt{IterMetrics} or \texttt{FMLP\_Metrics} class 
in the \href{https://github.com/gnthr-solve/TP_ML_Pipeline}{github repository} of our project.\\

If the predicted probabilities line up perfectly with the mean frequency they form a line matching the identity function.
A model that fulfils this is called well calibrated.
What seems counterintuitive at first sight is that a model can have an excellent AUC value but at the same time be poorly calibrated and vice versa.
An optimal model would have both good calibration and good discrimination, but this is difficult to achieve.
Especially for imbalanced datasets one often has to sacrifice one for the other.

\begin{comment}
\subsubsection{Decision Theoretic Measures}
Suppose the classifier is presented with mammography image data and the predicted probabilities for cancer in a case are $49\%$ for class $0$ (i.e. healthy tissue)
and $51\%$ for class $1$ (i.e. malignancy). The classifier would confidently predict a cancer diagnosis that may incur a risky biopsy for the patient.
Depending on the circumstances one might want to adapt or at least assess the probability threshold at which the classifier predicts class $1$.
To assess a classifiers performance in a context like this measures have been proposed that take into account the predicted probabilities and the risks involved.

\textbf{Risk Threshold}
Suppose r.v. $Y$ with outcome diseased $D$ or healthy $\neg D$.
Let $c_{TP}, c_{FP}, c_{TN}, c_{FN}$ then expected cost of predicting $\neg D$ is
\[
	\mathbb{P}(Y = 1) c_{FN} + \mathbb{P}(Y = 0) c_{TN}
\]

and expected cost for predicting $D$ is
\[
	\mathbb{P}(Y = 1) c_{TP} + \mathbb{P}(Y = 0) c_{FP}
\]

If we write $T = \mathbb{P}(Y = 1)$ we can rewrite to
\[
	T c_{FN} + (1-T) c_{TN}
\]
and
\[
	T c_{TP} + (1-T) c_{FP}
\]
One is indifferent about treatment if both expected costs are equal.
\[
\begin{aligned}
	T c_{FN} + (1-T) c_{TN} &= T c_{TP} + (1-T) c_{FP} \\
	&\Leftrightarrow \\
	T &= \frac{c_{TN} - c_{FP}}{(c_{TN} - c_{FP}) + (c_{TP} - c_{FN})} \\
\end{aligned}
\]
It can also be rearranged in a different way
\[
\begin{aligned}
	T c_{FN} + (1-T) c_{TN} &= T c_{TP} + (1-T) c_{FP} \\
	&\Leftrightarrow \\
	\frac{T}{1 - T} &= \frac{ c_{TN} - c_{FP} }{ c_{TP} - c_{FN} } \\
\end{aligned}
\]

In the first group, the costs relate to undertreatment (false-negative classifications)
The costs of these false-negative classifications $c_{FN}$ should be compared to the costs of true-positive classifications $c_{TP}$.
The difference $c_{TP} - c_{FN}$ is the net benefit of treating all who have the disease compared to treating none of them.
Suppose you have a treatment that causes lots of damage as well as curing the disease. 
Then this difference might be low, i.e. treating everyone with the dangerous treatment does not give much better outcome than simply not treating anyone.
Suppose on the other hand you have a devastating disease like Polio and a low cost treatment like a Polio-vaccine, then that difference will be strongly positive.

In the second group, relevant costs are for those without the event if not treated, who are treated (“overtreated”).
The costs of these false-positive classifications ($c_{FP}$) should be compared to the costs of true-negative classifications ($c_{TN}$)
while $c_{TN} - c_{FP}$ is the harm of treating all who don't have the disease compared to the benefit of not treating any of them.
E.g. the cost of not treating anyone without the disease might be 0 but the cost of treating them might be high. Then this value is strongly negative.
On the other hand suppose again a low impact vaccine that barely does harm, then that difference may be small negative. 

Odds (cutoff) = Harm/Benefit

The ratio in case of e.g. the polio vaccine could be strongly in favour of treating everyone (small cost for those without, high benefit for those with disease).


\textbf{Net Benefit}
The choice of risk threshold implicitly conveys the adopted relative misclassification costs. 
It can be derived that the odds of the risk threshold equal the harm-to-benefit ratio, which is the harm of a false positive divided by the benefit of a true positive.
For example, if a risk threshold of $20\%$ is used, the odds are 1 to 4. 
Therefore, a $20\%$ risk threshold assumes that the harm of a false positive is one-quarter of the benefit of a true positive or that 1 true positive is worth 4 false positives: 
A clinician might express this in terms such as ‘‘I would not do more than five biopsies to find one cancer.’’ 
Hence, when applying a model to a set of patients,
we can correct the number of true positives (TP) for the number of false positive (FP) using the odds $w$ of the risk threshold $t$: 
\[
	\text{TP} - w \text{FP} = \text{TP} - \frac{\tau}{1-\tau} \text{FP}
\]
When dividing by the total sample size $N$, the Net Benefit is obtained
\[
	\text{NB} = \frac{1}{N} (\text{TP} - w \text{FP}) = \frac{1}{N} (\text{TP} - \frac{\tau}{1-\tau} \text{FP})
\]

The Net Benefit of treat-none is always 0, 
whereas the Net Benefit of treat-all is positive for risk thresholds below the event rate and negative for risk thresholds above the event rate.
	
\end{comment}
	

\subsection{Hypothesis testing}
The \textbf{Hypothesis-T-Test} class is designed to facilitate hypothesis testing, specifically t-tests, on datasets stored in CSV format. It enables us to compare the effectiveness of different combinations of data balancing and classification methods in terms of key performance measures. The t-test is an appropriate choice in this case, where we want to apply multiple pairwise comparisons of continuous variables (the target measures). Hereby is an overview of its functionality.

The class constructor accepts several essential parameters: 
\begin{enumerate}[label=$\bullet$]
\item \textbf{target list}: a list of target performance measures;
\item \textbf{column bal}: the name of the file column addressing the data balancing method
\item \textbf{column clas}: the name of the file column addressing the classifier method
\item \textbf{test combination balancers}: a list of tuples, each containing a combination of two balancing methods for comparison
\item \textbf{test combination classifiers}: a list of tuples, each containing a combination of two classifier methods for comparison
\item \textbf{alpha}: the test significance level, set by default to 0.05, a typical choice in hypothesis testing. 
\end{enumerate}

The core functionality of the class resides in the \textbf{perform-t-test} method, which consists of the following steps:
\begin{enumerate}[label=$\bullet$]
\item Initialization of an empty list, \textbf{results}, to store the outcomes of the t-tests.
\item For each target performance measure, the class iterates through the provided combinations of balancing methods and classifier methods.
\item The data is filtered based on the selected balancing or classifier method, creating two distinct groups.
\item A t-test is conducted on the two groups, producing statistical values such as the t-statistic and p-value.
\item The mean values of the target performance measure are calculated for each group.
\item The test results, including the target performance measure, compared methods, mean values, t-statistic, and p-value, are organized into dictionaries and appended to the \textbf{results} list.
\item After completing all t-tests, the results are compiled into a pandas DataFrame and saved to a CSV file. This file serves as a comprehensive record of the statistical comparisons.
\end{enumerate}

\subsection{Linear regression Analysis}
Another crucial step in our pipeline is to thoroughly explore the relationships between our target measures and the parameters of interest that we allowed to vary in the preceding stages of our project. To achieve this, we've developed the \textbf{Linear Regression Analysis} class. This class empowers us to conduct in-depth investigations into how specific predictor variables influence our chosen target metrics. In the following, we will delve into the functionality of this class and showcase its applications in our data analysis process.
After loading a dataset from a specified CSV file and storing it in a pandas DataFrame, the data need to be set up for regression analysis. This is done through the \textbf{prepare data} method, which takes two sets of input variables: categorical and continuous regressors. The categorical variables are encoded into binary format and combined with the continuous regressors to create the input matrix \textbf{X}.
The actual analysis is then implemented in the function \textbf{perform linear regression}, which takes the following inputs: 
\begin{enumerate}[label=$\bullet$]
\item \textbf{target}: the target metric that we want to predict
\item \textbf{regressors}: a list of predictor variables used in the regression analysis 
\end{enumerate}
The function performs the following steps: 
\begin{enumerate}[label=$\bullet$]
\item Splits the dataset into training and testing sets.
\item Fits a linear regression model to the training data.
\item Predicts the target metric for the test data.
\item Calculates various regression metrics, including mean squared error, mean absolute error, coefficient values, coefficient of determination (R-squared), F-value, and p-value.
\item Stores the results in a pandas DataFrame and returns it.
\end{enumerate}

Finally, the method \textbf{plot target vs regressors} repeats the analysis and loops through each regressor in the regressors list. For each regressor, the corresponding predicted values \textbf{y values} are calculated as a sum of the intercept and the product between the regression coefficients and the data. Next, a scatter plot is produced to graphically observe the relationship.







\begin{thebibliography}{9}

\bibitem{SMOTE}
 Nitesh, V., Chawla., Kevin, W., Bowyer., Lawrence, O., Hall., W., Philip, Kegelmeyer. (2002). SMOTE: synthetic minority over-sampling technique. Journal of Artificial Intelligence Research, 16(1):321-357. doi: 10.1613/JAIR.953


\bibitem{ADASYN}
 Haibo He, Yang Bai, E. A. Garcia and Shutao Li, "ADASYN: Adaptive synthetic sampling approach for imbalanced learning," 2008 IEEE International Joint Conference on Neural Networks (IEEE World Congress on Computational Intelligence), Hong Kong, 2008, pp. 1322-1328, doi: 10.1109/IJCNN.2008.4633969.
\end{thebibliography}	
\end{document}