

\definecolor{myblue}{cmyk}{1,.72,0,.38}

% Define custom colors
%\definecolor{myblue}{rgb}{0,0,1}
\definecolor{myyellow}{rgb}{1,1,0}
\definecolor{myorange}{rgb}{1,0.5,0}
\definecolor{mypurple}{rgb}{0.5,0,0.5}
\definecolor{mygreen}{rgb}{0,0.5,0}

%Python code style settings---------------------------------------------------------------------------------------------------------------------------------------------------------------------------------------

% Define a custom style for Python code
\lstdefinestyle{mystyle}{
    backgroundcolor=\color{white},       % background color
    commentstyle=\color{mygreen},        % comment color
    keywordstyle=\color{mypurple},         % keyword color
    numberstyle=\tiny\color{gray},     % line numbers color
    stringstyle=\color{orange},       % string color 
    identifierstyle =\color{myblue},
    basicstyle=\ttfamily\scriptsize,          % code font and size
    breakatwhitespace=false,             % break lines only at whitespace
    breaklines=true,                     % enable line breaking
    captionpos=b,                        % caption position (bottom)
    keepspaces=true,                     % keep spaces in code
    numbers=left,                        % line numbers position (left)
    numbersep=5pt,                       % distance between line numbers and code
    %numbers=none,                        % line numbers position (left)
    showspaces=false,                    % show spaces using underscores
    showstringspaces=false,              % show spaces in strings as underscores
    showtabs=false,                      % show tabs using underscores
    tabsize=4                            % tab size
}

% Set custom colors for specific keywords and function names
\lstset{
    language=Python,
    style=mystyle,
    %morekeywords={class, def},   % Add keywords to highlight
    %keywordstyle=\color{myblue}, % Keyword color
    %morekeywords={for, in},
    %keywordstyle=\color{mypurple}, % Keywords "for" and "in" will be purple       
    emph={max, min, sum},                  % Define function name(s) to highlight
    emphstyle=\color{yellow},          % Function name color
    %emph={for, in},                  % Define function name(s) to highlight
    %emphstyle=\color{mypurple},          % Function name color
}



%Own Operators----------------------------------------------------------------------------------------------------------------------------------------------------------------------------------------------------
\DeclareMathOperator*{\argmax}{arg\,max}
\DeclareMathOperator*{\argmin}{arg\,min}

\DeclareMathOperator*{\curl}{curl}

\let\div\relax %remove the existing div command to allow operator definition
\DeclareMathOperator*{\div}{div}

\DeclareMathOperator*{\grad}{grad}



% Redefining single $...$ inline math------------------------------------------------------------------------------------------------------------------------------------------------------------------
\let\originalmathdollar=$
\catcode`\$=\active
\def$#1${\originalmathdollar\color{myblue}#1\originalmathdollar}

% Redefining the display math environment ------------------------------------------------------------------------------------------------------------------------------------------------------------------
% Save the original definition of the display math environment
\let\originaldisplaymath=\[
\let\endoriginaldisplaymath=\]

% Redefine the display math environment to include color
\renewcommand{\[}{\begin{originaldisplaymath}\color{myblue}}
\renewcommand{\]}{\end{originaldisplaymath}}



% Redefining the equation environment ------------------------------------------------------------------------------------------------------------------------------------------------------------------
% Save the original definition of the equation environment
\let\originalequation=\equation
\let\endoriginalequation=\endequation

% Redefine the equation environment to include color
\renewenvironment{equation}{\begin{originalequation}\color{myblue}}{\end{originalequation}}









